\documentclass{article}
\usepackage{Style}
\title{Differential Equations Week 2}
\author{Garud Shah}
\begin{document}
\maketitle
\begin{problem}[Problem 1a]
    Consider the differential equation $y' = 3x^2y^2$.
        \begin{enumerate}[I.]
            \item Prove that there is a unique solution on some interval around all initial values.
            \item Draw at least 5 level curves of $y'(x,y)$.
            \item Solve the differential equation.
            \item Draw a direction field for the differential equation.
            \item For the inital value problem $y(0)=1$:
            \begin{enumerate}[1.]
                \item Draw the solutions of the differential equation with this 
                intial value onto the direction field.
                \item Approximate the solutions of the differential equation with Euler's Method,
                with a delta-value of $\Delta$. 
                \item Graph the solutions in 1aV2 for
                \begin{enumerate}[(a)]
                    \item $\Delta = 0.4$
                    \item $\Delta = 0.2$
                    \item $\Delta = 0.01$.
                \end{enumerate}
            \end{enumerate}
            \item For the inital value problem $y(0)=0$:
            \begin{enumerate}[1.]
                \item Draw the solutions of the differential equation with this 
                intial value onto the direction field.
                \item Approximate the solutions of the differential equation with Euler's Method,
                with a delta-value of $\Delta$. 
                \item Graph the solutions in 1aVI2 for
                \begin{enumerate}[(a)]
                    \item $\Delta = 0.4$
                    \item $\Delta = 0.2$
                    \item $\Delta = 0.01$.
                \end{enumerate}
            \end{enumerate}
        \end{enumerate}
    \end{problem} 
    \begin{problem}[Problem 1b]
    Consider the differential equation $y' = \sqrt[3]{y}$.
        \begin{enumerate}[I.]
            \item Prove that there is a unique solution on some interval around all initial values if $y_0 \ne 0$.
            \item Draw at least 5 level curves of $y'(x,y)$.
            \item Solve the differential equation.
            \item Draw a direction field for the differential equation.
            \item For the inital value problem $y(0)=1$:
            \begin{enumerate}[1.]
                \item Draw the solutions of the differential equation with this 
                intial value onto the direction field.
                \item Approximate the solutions of the differential equation with Euler's Method,
                with a delta-value of $\Delta$. 
                \item Graph the solutions in 1bV2 for
                \begin{enumerate}[(a)]
                    \item $\Delta = 0.4$
                    \item $\Delta = 0.2$
                    \item $\Delta = 0.01$.
                \end{enumerate}
            \end{enumerate}
            \item For the inital value problem $y(0)=0$:
            \begin{enumerate}[1.]
                \item Draw the solutions of the differential equation with this 
                intial value onto the direction field.
                \item Approximate the solutions of the differential equation with Euler's Method,
                with a delta-value of $\Delta$. 
                \item Graph the solutions in 1bVI2 for
                \begin{enumerate}[(a)]
                    \item $\Delta = 0.4$
                    \item $\Delta = 0.2$
                    \item $\Delta = 0.01$.
                \end{enumerate}
            \end{enumerate}
        \end{enumerate}
\end{problem}
\begin{solution}[Problem 1aI]
    Notice the following:
    \begin{itemize}
        \item $f(x,y)$ is continuous everywhere.
        \item $\dfrac{\partial f}{\partial y} = 6x^2y$ is also continous everywhere.
        \item By existence and uniqueness don.
    \end{itemize}
\end{solution}
\begin{solution}[Problem 1aII]
    BLAHS
\end{solution}
\begin{solution}[Problem 1bI]
    Notice the following:
    \begin{itemize}
        \item $f(x,y)$ is continuous everywhere.
        \item $\dfrac{\partial f}{\partial y} = \dfrac{1}{3y^{2/3}}$ is continous everywhere except $y=0$.
        \item By existence and uniqueness done.
    \end{itemize}
\end{solution}
\begin{problem}[Problem 3]
    
\end{problem}
\begin{solution*}
    
\end{solution*}
\begin{problem}[Problem 5 (Equivilant to 2.3 39)]
    Consider the following initial value problem:
    \begin{align}
        T(0) &= 0; \\ 
        \dfrac{dT}{dt} &= -T + 
        \begin{cases}
            0, \lfloor t \rfloor \text{ is odd} \\ 
            1, \lfloor t \rfloor \text{ is even.}
        \end{cases}
    \end{align}
    \begin{enumerate}[(a)]
        \item Find $100T(3)$ rounded to the nearest whole number.
        \item Find $100T(8)$ rounded to the nearest whole number.
    \end{enumerate}
\end{problem}
\begin{solution*}
    \begin{lemma}
        If $t$ is an odd whole number:
        \begin{align*}
            T(t) = \sum_{i=0}^{t} \dfrac{(-1)^i}{e^i}.
        \end{align*}
    \end{lemma}
    \begin{proof}
        First, we solve the relavent differential equationL
        \begin{align*}
            \dfrac{dT}{dt} &= 1- T \\ 
            \dfrac{T'}{1-T} &= 1 \\ 
            \int \dfrac{1}{1-T} dT &= t \\
            -\log |1-T| &= t + C \\ 
            1-T &= ce^{-t} \\ 
            T &= 1-ce^{-t}.
        \end{align*}
        Then, proceed by induction. \\~\\ 
        \IHID $t-1$ case of (Lemma 5.2)/$t-2$ case of this lemma (equivilant as noted in the lemma, ** of 5.2 provides us our
        base case) \\ 
        \ISID Note that the recurrence is $T(t) = 1 + \dfrac{1}{e}(T(t-1)-1)$. \\~\\ 
        Working out the algebra, we get 
        \begin{align*}
            &1-\dfrac{1}{e} + \sum_{i=2}^{t} \dfrac{(-1)^i}{e^i} \\ 
            =&\sum_{i=0}^{t} \dfrac{(-1)^i}{e^i},
        \end{align*}
        and we are done with this lemma. \\~\\ 
        *note the case where $T(t) = 1$: however this gets factored back in at the end as $c$ can be any value
    \end{proof}
    \begin{lemma}
        If $t$ is an even whole number:
        \begin{align*}
            T(t) = \sum_{i=1}^{t} \dfrac{(-1)^{i+1}}{e^i}.
        \end{align*}
    \end{lemma}
    \begin{proof}
        When $\lfloor t \rfloor$ is ODD, notice that:
        \begin{align*}
            \dfrac{dT}{dt} &= -T \\ 
            \dfrac{T'}{T}* &= -1 \\
            \int \dfrac{1}{T} dT &= -t \\ 
            - \log \mid T\mid &= -t + C \\ 
            T &= ce^{-t} *,
        \end{align*}
        for $t-1\le T\le t$. This implies $T(t) = \dfrac{T(t-1)}{e}$ which** is the above formula given that (Lemma 5.1)
        holds for $t-1$. \\
        *note the case where $T(t) = 0$: however this gets factored back in at the end as $c$ can be any value
        **unless $t=0$, when this statement trivally holds as empty sums are zero
    \end{proof}
    
\end{solution*}
\end{document}