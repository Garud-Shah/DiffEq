\documentclass[11pt]{article}
\usepackage{gstyle}
\title{Differential Equations Week \textbf{14}}
\author{Garud Shah}
\begin{document}
    \maketitle \newpage 
    \begin{problem}
        Show that:
        \begin{align}
            \Gamma(x+1) = x\Gamma(x),
        \end{align}
        for all $x \in \RR^+$.
    \end{problem}
    \begin{solution}
        Note that:
        \begin{align}
            \Gamma (x) &= \int _{0}^{\infty }t^{x-1}e^{-x}dx \\
            &= \LP[t^{x-1}](1).
        \end{align}
        As:
        \begin{align}
            \Gamma(x+1) &= \LP[xt^{x-1}] \\ 
            &= \LP[(t^{x})'] \\
            &= s \cdot \LP[xt^{x-1}] + (t^x)'(0) \\
            &= s \cdot \LP[xt^{x-1}] + 0 \\
            &= x\Gamma(x).
        \end{align}
    \end{solution}
    \newpage
    \begin{problem}
        Compute the following Laplace transforms:
        \begin{enumerate}[(a)]
            \item $\LP [t \cos bt]$
            \item $\LP [t^2 \cos bt]$
        \end{enumerate}
    \end{problem}
    \begin{solution} Note that 
        \begin{align}
            \LP [t^n f(t)] &= (-D)^nF(s) \\
            \LP [\cos bt] &= \dfrac{s}{b^2+s^2} \\
            \LP [t \cos bt] &= -D \left(\dfrac{s}{b^2+s^2}\right) \\
            &= -\dfrac{b^2+s^2-s(2s)}{(b^2 + s^2)^2} \\
            &= \dfrac{s^2-b^2}{(b^2+s^2)^2}.
        \end{align}
    \end{solution}
    \begin{solution} Note that 
        \begin{align}
            \LP [t^n f(t)] &= (-D)^nF(s) \\
            \LP [t \cos bt] &= \dfrac{s^2-b^2}{(b^2+s^2)^2} \\
            \LP [t^2 \cos bt] &= -D \left(\dfrac{s^2-b^2}{(b^2+s^2)^2}\right) \\
            &= -\dfrac{2s(b^2+s^2)^2-4s(s^2-b^2)(s^2+b^2)}{(b^2 + s^2)^4} \\
            &= -\dfrac{2sb^4+4s^3b^2 +2s^5 - 4s^5 +4sb^4}{(b^2+s^2)^4} \\
            &= -\dfrac{-2s^5+4s^3b^2+6sb^4}{(b^2+s^2)^4} \\
            &= \dfrac{2s(s^4-2s^2b^2-3b^4)}{(b^2+s^2)^4} \\
            &= \dfrac{2s(s^2+b^2)(s^2-3b^2)}{(b^2+s^2)^4} \\
            &= \dfrac{2s(s^2-3b^2)}{(b^2+s^2)^3}.
        \end{align}
    \end{solution}
    \newpage
    \begin{problem}
        Show that the portion of the partial fraction expansion of \\ $\dfrac{P(s)}{(s-r_i)(s-r_1)(s-r_2) \cdots} = \dfrac{P(x)}{Q(x)}$ is:
        \begin{align}
            \dfrac{P(r)}{Q'(r)}.
        \end{align}
        and that:
        \begin{align}
            \LP^{-1} \left[\dfrac{P}{Q}\right] &= \sum_{i=1}^{\deg Q} \dfrac{P(r_i)}{Q'(r_i)}e^{r_it}
        \end{align}
    \end{problem}
    \begin{solution}
        Note that evaluating at $s=r_i + \varepsilon$, we have:
        \begin{align}
            \dfrac{P(s)}{(s-r_i)(s-r_1)(s-r_2) \cdots } &= \sum_{n=1}^{\deg Q} \dfrac{A_n}{s-r_n} \\
            \dfrac{P(s)}{\varepsilon \cdot (s-r_1)(s-r_2) \cdots } &= \dfrac{A_i}{\varepsilon} + \sum_{n=1, n \ne i}^{\deg Q} \dfrac{A_n}{r_i+\varepsilon-r_n} \\
            \dfrac{P(s)}{Q(s)/(s - r_i)} &= A_i + \varepsilon \cdot \sum_{n=1, n \ne i}^{\deg Q} \dfrac{A_n}{s-r_n}.
        \end{align}
        Now take $\displaystyle \lim_{\varepsilon \rightarrow 0}$ to get:
        \begin{align}
            \lim_{\varepsilon \rightarrow 0} \dfrac{P(s)}{Q(s)/(s - r_i)} &= \lim_{\varepsilon \rightarrow 0} A_i + \varepsilon \cdot \sum_{n=1, n \ne i}^{\deg Q} \dfrac{A_n}{s-r_n} \\
            \lim_{\varepsilon \rightarrow 0} \dfrac{P(s)}{Q(s)/(s - r_i)} &= A_i  \\
            r &= r_i \\
            \lim_{s \rightarrow r} \dfrac{P(s)(s-r)}{Q(s)} &= A_i,
        \end{align}
        and this is $0/0$ L'Hopital. Taking derivitatives, we get:
        \begin{align}
            R(\alpha) &= P(r+\alpha) \\
            \lim_{s \rightarrow r} \dfrac{P(s)(s-r)}{Q(s)} &= \lim_{\varepsilon \rightarrow 0} \dfrac{\varepsilon R(\varepsilon)}{Q(r+\varepsilon)} \\
            &= \lim_{\varepsilon \rightarrow 0} \dfrac{R(\varepsilon) + \varepsilon R'(\varepsilon)}{Q'(r+\varepsilon)} \\
            &= \lim_{\varepsilon \rightarrow 0} \dfrac{R(\varepsilon)}{Q'(r+\varepsilon)}  +  \lim_{\varepsilon \rightarrow 0} \dfrac{\varepsilon R'(\varepsilon)}{Q'(s)}  \\
            |Q'(r_i)| &> 0 \text{ (if it was 0, }r_i\text{ would be a double root)} \\
            &= \dfrac{P(r_i)}{Q'(r_i)}.
        \end{align}
        Now, noting that:
        \begin{align}
            \LP^{-1}\left[\dfrac{1}{s-r_i}\right] &= e^{r_it},
        \end{align}
        using the previous finding, and summing over all $r_i$, and using linearity of $\LP^{-1}$, we get:
        \begin{align}
            \boxed{\dfrac{P}{Q} = \sum_{i=1}^{\deg Q} \dfrac{P(r_i)}{Q'(r_i)(s-r_i)}}
        \end{align}
        \begin{align}
            \boxed{\LP^{-1} \left[\dfrac{P}{Q}\right] = \sum_{i=1}^{\deg Q} \dfrac{P(r_i)}{Q'(r_i)}e^{r_it}}
        \end{align}
    \end{solution}
    \newpage
    \begin{problem}
        Find the partial fraction decomposition of:
        \begin{align}
            \dfrac{2s+1}{s(s-1)(s+2)}
        \end{align}
    \end{problem}
    \begin{solution}
        Use (3.17). \\
        There are two things to start with:
        \begin{itemize}
            \item Find $Q'$.
            \item Find roots of $Q$.
        \end{itemize}
        Note that roots of $Q$ are $-2,0,1$. Now also note that $Q' = (s^3+s^2-2s)' = 3s^2+2s-2$ and plugging in:
        \begin{align}
            \dfrac{2s+1}{s(s-1)(s+2)} &= \dfrac{P(-2)}{Q'(-2)} \cdot \dfrac{1}{s+2} + \dfrac{P(0)}{Q'(0)} \cdot \dfrac{1}{s} + \dfrac{P(1)}{Q'(1)} \cdot \dfrac{1}{s-1} \\
            &= -\dfrac{1}{2} \cdot \dfrac{1}{s+2} + \dfrac{1}{3} \cdot \dfrac{1}{s} + \dfrac{1}{s-1}.
        \end{align}
    \end{solution}
    \begin{problem}
        Find the inverse Laplace transform of:
        \begin{align}
            \dfrac{3s^2-16s+5}{(s+1)(s-2)(s-3)}
        \end{align}
    \end{problem}
    \begin{solution}
        Use (3.18). \\
        There are two things to start with:
        \begin{itemize}
            \item Find $Q'$.
            \item Find roots of $Q$.
        \end{itemize}
        Note that roots of $Q$ are $-2,0,1$. Now also note that $Q' = (s^3-4s^2+s-6)' = 3s^2-8s+1$ and plugging in:
        \begin{align}
            \LP^{-1} \left[\dfrac{3s^2-16s+5}{(s+1)(s-2)(s-3)}\right] &= \dfrac{P(-1)}{Q'(-1)} \cdot e^{-t} + \dfrac{P(2)}{Q'(2)} \cdot e^{2t} + \dfrac{P(3)}{Q'(3)} \cdot e^{3t} \\
            &= 2e^{-t}-3e^{2t} -13 \cdot e^{3t} \\
        \end{align}
    \end{solution}
    \newpage 
    \begin{problem}
        Solve the intial value problem:
        \begin{align}
            y''(t) + 4y &= 
            \begin{cases}
            3\sin t, &0\le t \le 2\pi \\ 
            0, &2\pi<t   \end{cases} \\
            &\begin{cases}
                y(0) = 1 \\
                y'(0) = 3
            \end{cases}
        \end{align}
    \end{problem}
    \begin{solution}
        \textbf{Step 1.} Express the piecewise continuous forcing function in terms of heaviside functions. \\
        Note that for any piecewise continuous function that is equal to $f_k$ on intereval $I_k = (a_k,b_k)$, that function is equal to:
        \begin{align}
            \sum_{I_k} f_k(x)u(x-a_k) - f_k(x)u(x-b_k),
        \end{align}
        and note that for our forcing function, that is:
        \begin{align}
            \boxed{3\sin t - 3\sin t \cdot u(x-2\pi).}
        \end{align}
    \end{solution}
    \begin{solution}
        \textbf{Step 2.} Take the Laplace Transform and solve for $\LP[y]$. \\
        Taking the Laplace transform, we get:
        \begin{align}
            \LP[y''(t) + 4y] &= \LP[3\sin t - 3\sin t \cdot u(x-2\pi)] \\
            (s^2+4)\LP[y] - sy(0)-y'(0) &= \dfrac{3(1-e^{2\pi s})}{1+s^2}
        \end{align}
        \begin{align}
            \boxed{\LP[y] = \dfrac{3}{(1+s^2)(4+s^2)} - \dfrac{3e^{2\pi s}}{(1+s^2)(4+s^2)}+ \dfrac{s-3}{s^2+4}}
        \end{align}
    \end{solution}
    \newpage
    \begin{solution}
        \textbf{Step 3.} Take the Inverse Laplace Transform and solve for $u$. \\
        Taking the inverse Laplace transform, we get:
        \begin{align}
            \LP[y] &= \dfrac{3}{(1+s^2)(4+s^2)} - \dfrac{3e^{2\pi s}}{(1+s^2)(4+s^2)}+ \dfrac{s-3}{s^2+4} \\
            y&= \LP^{-1} \left[\dfrac{3}{(1+s^2)(4+s^2)} - \dfrac{3e^{2\pi s}}{(1+s^2)(4+s^2)}+ \dfrac{s-3}{s^2+4}\right] \\
            &= \cos(2t) - \dfrac{3}{2}\sin(2t) + 3(1-u(t-2\pi))\LP^{-1} \left[\dfrac{3}{(1+s^2)(4+s^2)} \right] \\
            &= \cos(2t) - \dfrac{3}{2}\sin(2t) \\
            &+ 3(1-u(t-2\pi))\LP^{-1} \left[\dfrac{1}{(1+s^2)}-\dfrac{1}{(4+s^2)} \right] \\
            &= \boxed{3(1-u(t-2\pi))\left(\sin(t)-\dfrac{1}{2}\sin(2t)\right) + \cos(2t) - \dfrac{3}{2}\sin(2t).}
        \end{align}
    \end{solution}
\end{document}