\documentclass{beamer}
\usepackage{graphicx}
\AtBeginSection[]{
    \begin{frame}
        \frametitle{Table of Contents}
        \tableofcontents[currentsection]
    \end{frame}
}
\AtBeginSection[]
{
  \begin{frame}
    \frametitle{Table of Contents}
    \tableofcontents[currentsection]
  \end{frame}
}
%Information to be included in the title page:
\title[Climate Change Differential Equations]{\normalsize A Brief Model Of \\
\Huge Climate Change}
\author{Garud S}
\date{March 17th, 2025}
\usetheme{Antibes}
\begin{document}
\frame{\titlepage}
\section{What does the future hold?}
    \begin{frame}\frametitle{Climate Change}
        It's a thing, turns out. \\
        But how bad is it? For that we need to consider models. \\
    \end{frame}
    \begin{frame}\frametitle{Modeling Parameters}
        There are three factors: \\
        $G_C$, carbon-containg greenhouse gases (this is what we effect) \\
        $T$, global average temprature \\ 
        $G_O$, basically water whose concentration is effected by heat. This will fill up to roughly 50\% concentration.
    \end{frame}
\section{Mathematical Modeling}
    \begin{frame}\frametitle{Model: CO$_2$}
        Let $h(t)$ be the non-natural addition/removal (this is what ``net zero'' is about) of greenhouse gases into the atmospeher by us. Let $N$ be the natural addition of carbon-containg greenhouse gases. How do we model natural removal?
        We model the removal with linear approximation, with removal capacity being $G_Cc$ where $c$ is a constant. (Any constant removal is factored into $N$, 
        so we replace $N$ with $\Delta_N$.) Note that at standard GHG level, $S$, with $u(t)=0$, $G_C'=0$ and $-cS + \Delta_N=0$ so $c = \dfrac{\Delta_N}{S}$.
        So:
        \begin{align}
            G_C' &= h(t) + \Delta_N - \dfrac{G_C\Delta_N}{S}.
        \end{align}
    \end{frame}
    \begin{frame}\frametitle{Model: $T$}
        We know that heat comes in naturally, and it leaves naturally. This term is $\Delta_T$. However, $\gamma_C G_C + \gamma_O G_0$
        amount of heat is reflected back. So,
        \begin{align}
            T' &= \Delta_T + \gamma_C G_C + \gamma_O G_O.
        \end{align}
        (Heat is in celsius!)
    \end{frame}
    \begin{frame}\frametitle{Model: Water}
        This is directly related to temprature, so it's just the capacity that's filled. That is:
        \begin{align}
            G_O &= cT + K.
        \end{align}
    \end{frame}
    \begin{frame}\frametitle{Wait-This-Can-Be-Solved}
        It turns out \textit{water can be just plugged in, carbon dioxide is just a 1st order linear, 
        so temprature is just a 1st order linear!}
    \end{frame}
\section{Solving The Differential Equation}
\begin{frame}\frametitle{Carbon Dioxide}
    \begin{align}
        G_C' &= h(t) + \Delta_N - \dfrac{G_C\Delta_N}{S}
    \end{align}
    $\text{ }$\\~\\~\\~\\~\\
\end{frame}
\begin{frame}\frametitle{Temprature}
    \begin{align}
        T' &= \Delta_T + \gamma_C G_C + \gamma_O (cT + K).   
    \end{align}
    $\text{ }$\\~\\~\\~\\~\\
\end{frame}
\begin{frame}\frametitle{Desmos Graphs}
    
\end{frame}
    \section{End}
        \begin{frame}\frametitle{Questions}
            Questions? \\~\\~\\~\\
        \end{frame}
        \begin{frame}
            \begin{center}
                Thanks for listening!
            \end{center}
        \end{frame}
\end{document} 