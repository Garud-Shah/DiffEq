\documentclass{article}
\usepackage{sty}
\title{Differential Equations Week 1}
\author{Garud Shah}
\begin{document}
\maketitle
\newpage
\textbf{Bold} indicates corrections.
\begin{problem}
    Let $c>0$. Show that $\phi(x) = \dfrac{1}{c^2-x^2}$ is a solution to the inital value problem
    $y' = 2xy^2$, $y(0) = \dfrac{1}{c^2}$ on the interval $-c < x < c$.
\end{problem}
\begin{solution*}
    Plugging in, we get:
    \begin{align*}
        & \left(\dfrac{1}{c^2-x^2}\right)' & \text{start}\\ 
        &= -2x \cdot -\left(\dfrac{1}{c^2-x^2}\right)^2 & \text{chain rule}\\ 
        &= 2xy^2. & \text{finish \done}
    \end{align*}
\end{solution*}
\begin{problem}
    Consider the nonlinear differential equation $\dfrac{dy}{dx}=3y^{2/3}$ for all $x$.
    \begin{enumerate}[(a)]
    \item Verify that all functions of the form $f(x)={(x-c)}^3$ are explicit solutions to the equation for all $x$. 
    \item Verify that the identically zero function $f(x)=0$ also satisifies the equation for all $x$.
    \item Verify that the function defined by 
    $f(x) = \begin{cases}
    {(x-c)}^3, & x>c \\ 
    0, &x \le c\end{cases}$ satisfies the equation for all $x$, and that $\begin{cases}
    {(x-c)}^3, & x\le c \\ 
    0, &x > c\end{cases}$ does too.
    \item Is there any point $(a,b)$
    in the plane such that there is no solution to the differential equation that passes through that point?
    \item Is there any point $(a,b)$
    in the plane such that there is a unique solution to the differential equation that passes through that point?
    \end{enumerate}
\end{problem}
\hfill \break
\begin{solution}[Problem 2a]
    Plugging in:
    \begin{align*}
        &f'(x)   & \text{inital}\\ 
        &= 3(x-c)^2 & \text{chain rule} \\ 
        &= 3f(x)^{2/3} & \text{manipulating \done}.
    \end{align*}
\end{solution}
\begin{solution}[Problem 2b]
    Plugging in:
    \begin{align*}
        &f'(x)   & \text{inital}\\ 
        &= 0 & \text{duh} \\ 
        &= 3f(x)^{2/3} & \text{duh \done}.
    \end{align*}
\end{solution}
\begin{solution}[Problem 2c]
    As this is an first order differential equation, we only need to look at the function and its derivative. The only place where we could
    have a problem is where we switch functions (as in (a) and (b) looking at a section of those functions and combining do the other parts 
    for us) but notice that here the value and slope are both always zero and $0 = 3 \cdot 0^{2/3}$ \done
\end{solution}
\begin{solution}[Problem 2d]
    \no$\text{ }$Take a point $a,b$. We will only consider solutions of the form $(x-c)^3$. Then we have: $b = (a-c)^3$,
    and $c = a - \sqrt[3]{b}$, thus there is at least one of this form. \done
\end{solution}
\begin{solution}[Problem 2e]
    \no$\text{ }$The solution derived in problem 2d can be duplicated as each solution generates another solution with $y \ge 0$:
    \begin{align*}
        f(x) = 
        \begin{cases}
            {(x-c)}^3, & x>c \\ 
            0, &x \le c
        \end{cases}   
    \end{align*}
    and one with $y \le 0$ 
    \begin{align*}
        f(x) = 
        \begin{cases}
            0, & x\ge c \\ 
            {(x-c)}^3, &x <c
        \end{cases}   
    \end{align*} 
    as well from problem 2c. Thus, if $y>0$, we can add the $(x-c)^3$ solution and the $y \ge 0$ solution above, and a similar thing for $y < 0$. 
    At $y=0$ we have $(x-x_0)^3$ and $y=0$ so we are done \done
\end{solution}
\begin{problem}
    Consider the differential equation 
    \begin{align*}
        \dfrac{dy}{dx} = - \dfrac{1 + ye^{xy}}{1+xe^{xy}},
    \end{align*}
    \begin{enumerate}[(a)]
    \item Show that $x + y + e^{xy} = 0$ defines $y$ as an implicit function of $x$ on some interval containing $(-1, 0)$.
    \item Given that any relation of the form $x+y+e^{xy}=C$ for any real number $C$
    satisfies the given differential equation, when and around which points will we not have a guarentee by the implicit function theorem not guarentee a solution
    for us?
    \end{enumerate}
\end{problem}
\begin{solution}[Problem 3a]
    Notice that if we let $G(x,y) = x + y + e^{xy}$,
    \begin{align*}
        \dfrac{dy}{dx} = - \dfrac{\partial G / \partial x}{\partial G / \partial y}.
    \end{align*}
        By the implicit function theorem, as $1 \ne 0$ and $G(x, y) = 0$, we are done with (a). 
\end{solution}
\begin{solution}[Problem 3b]
    Notice that with $G(x,y) =x + y + e^{xy} -C$, we'd only fail if $1+xe^{xy} = 0$, by the 
    implicit function theorem, or $xy = - \log x$, $y = -\dfrac{\log x}{x}$. This would mean $x -\dfrac{\log x}{x} + e^{- \log x} = x - \dfrac{\log x - 1}{x} = C$,
    which is possible for all $x> 0$ so we are done.
\end{solution}
\begin{problem}
    Consider $\dfrac{dy}{dx}=3y^{2/3}$ again.
    \begin{enumerate}[(a)]
        \item Where does the existence and uniqueness theorem guarentee a unique solution to this
        differential equation?
        \item How do you reconsile this with the result of problem 2?
    \end{enumerate}
\end{problem}
\begin{solution}[Problem 4a]
    Notice that $f=3y^{2/3}$ is continuous everywhere, and $\dfrac{\partial f}{\partial y} = 2y^{-1/3}$ is continuous everywhere except
    $y=0$. So we should be guarenteed unique solutions everywhere except $y=0$. 
\end{solution}
\begin{solution}[Problem 4b]
    \textbf{It only guarentees on a small interval around the point for $y \ne 0$, which IS true.}
\end{solution}
\begin{problem}[Problem 5]
    Use the conversion of the initial value problem $P =(x_0, y_0)$ and $\dfrac{dy}{dx} = f(x,y)$ into the integral operator
    \begin{align}
        \begin{cases}
            \hat{O_{f,P}}[y(x)] = y_0 + \int_{x_0}^{x} f(t,y(t)) dt  \\ 
            \hat{O_{f,P}}[y(x)] = y(x).
        \end{cases}
    \end{align}
    \begin{enumerate}[(a)]
        \item Prove that if $f(x,y) = y$, and $P = (0,1)$, then 
        \begin{align*}
            \hat{O}_{f,P}^n[1] &= \sum_{i=0}^n \dfrac{x^i}{i!}.
        \end{align*}
        \item If $f(x,y) = 3x - y^2$, and $P= (0,0)$, find:
        \begin{align*}
            \hat{O}_{f,P}^3[0].
        \end{align*}
        \item If $f(x,y) = 3y^{2/3}$, and $P = $
    \end{enumerate}
\end{problem}
\begin{solution}[Problem 5a]
    Note that we just need:
    \begin{align*}
        1 + \int_{0}^{x} \sum_{i=0}^{n} \dfrac{x^i}{i!} dt &= \sum_{i=0}^{n+1} \dfrac{x^i}{i!},
    \end{align*}
    which is the case termwise integrating.
\end{solution}
\begin{solution}[Problem 5b]
    Notice that we need to find:
    \begin{align*}
        \int_0^x 3x - \left(\int_0^x 3x - \left(\int_0^x 3x dx \right)^2 dx \right)^2 dx.
    \end{align*}
    Note that this is:
    \begin{align*}
        &\int_0^x 3x - \left(\int_0^x 3x - \left(\int_0^x 3x dx \right)^2 dx \right)^2 dx \\ 
        &= \int_0^x 3x - \left(\int_0^x 3x - \left(\dfrac{3x^2}{2} \right)^2 dx \right)^2 dx \\ 
        &= \int_0^x 3x - \left(\int_0^x 3x - \dfrac{9x^4}{4} dx \right)^2 dx \\ 
        &= \int_0^x 3x - \left(\dfrac{3x^2}{2} - \dfrac{9x^5}{20} dx \right)^2 dx \\
        &= \int_0^x 3x - \left(\dfrac{9x^{4}}{4}-\dfrac{27x^{7}}{20}+\dfrac{81x^{10}}{400}\right) dx \\
        &= \dfrac{3x^2}{2} - \dfrac{9x^5}{20} + \dfrac{27x^8}{160} - \dfrac{81x^{11}}{4400}. 
    \end{align*}
\end{solution}
\end{document}